% Template for ICIP-2013 paper; to be used with:
%          spconf.sty  - ICASSP/ICIP LaTeX style file, and
%          IEEEbib.bst - IEEE bibliography style file.
% --------------------------------------------------------------------------
\documentclass{article}
\usepackage{spconf,amsmath,graphicx}

% Example definitions.
% --------------------
\def\x{{\mathbf x}}
\def\L{{\cal L}}

% Title.
% ------
\title{Advantages of Robust Principal Component Analysis over Gaussian Mixture Models for Background Subtraction of Noisy Videos}
%
% Single address.
% ---------------
\name{Pooya R Khorrami, Xingqian Xu, Mert Dikmen, Thomas S. Huang\thanks{Thanks to XYZ agency for funding.}}
\address{University of Illinois - Urbana Champaign \\ 
              Department of Electrical and Computer Engineering\\
              Beckman Institute,  405 N. Matthews Avenue, Urbana IL, 61801}
%
% For example:
% ------------
%\address{School\\
%	Department\\
%	Address}
%
% Two addresses (uncomment and modify for two-address case).
% ----------------------------------------------------------
%\twoauthors
%  {A. Author-one, B. Author-two\sthanks{Thanks to XYZ agency for funding.}}
%	{School A-B\\
%	Department A-B\\
%	Address A-B}
%  {C. Author-three, D. Author-four\sthanks{The fourth author performed the work
%	while at ...}}
%	{School C-D\\
%	Department C-D\\
%	Address C-D}
%
\begin{document}
%\ninept
%
\maketitle
%
\begin{abstract}
The abstract should appear at the top of the left-hand column of text, about
0.5 inch (12 mm) below the title area and no more than 3.125 inches (80 mm) in
length.  Leave a 0.5 inch (12 mm) space between the end of the abstract and the
beginning of the main text.  The abstract should contain about 100 to 150
words, and should be identical to the abstract text submitted electronically
along with the paper cover sheet.  All manuscripts must be in English, printed
in black ink.
\end{abstract}
%
\begin{keywords}
One, two, three, four, five
\end{keywords}
%

%\vspace{0.5in}

\section{Introduction}
%\label{sec:intro}
In the field of video surveillance, the user typically wishes to extract meaningful and salient information from a video sequence in a completely automatic fashion. In several instances, the video sequences are captured using stationary cameras leading to a relatively static scene layout. The absence of camera motion implies that the background of the video sequence exhibits very little variation while dynamic changes in the scene represent the objects of interest. In such a case, the most common approach is to perform background subtraction to separate said dynamic regions (i.e. foreground) from the background of each frame.

When performing background subtraction, some na\"ive methods include frame differencing and approximate median. %Frame difference simply subtracts the current frame from the previous frame and the pixels where the difference exceeds a threshold are marked as foreground. The approximate median technique, on the other hand, models the background as the median of several past frames in the sequence. This median image is then subtracted from subsequent frames and pixels whose difference exceed a threshold are marked foreground. 
While these algorithms are simple to use and quite efficient, the most popular technique, by far, is adaptive  Gaussian Mixture Models (GMMs) \cite{FriedmanGMM}. Friedman et al. posit that each pixel in the background image can be represented by a probability distribution formed by a mixture of Gaussians. If a pixel greatly deviates from its corresponding model, then the pixel is labeled foreground. While the number of Gaussians used at each pixel is usually fixed, there has been some work \cite{ZivGMM} that adaptively selects the number of mixture components.

Although background subtraction via Gaussian Mixture Models enjoys widespread use in the computer vision community, it is not without drawbacks. As opposed to the frame differencing and approximate median techniques, GMMs possess several parameters that must be individually tuned. This implies that the algorithm is innately sensitive to different scene configurations. Therefore it should be no surprise that GMMs tend to perform rather poorly on noisy videos where the foreground objects are not immediately distinguishable. 

When dealing with noisy video sequences, we advocate the use of Robust Principal Component Analysis (RPCA or Robust PCA) for background subtraction. Robust PCA \cite{RPCA09} refers to how any matrix M can be represented as the sum of a low-rank matrix L and sparse matrix S by solving a convex optimization problem. If we form the matrix M by stacking each frame of the video sequence as a column, then we will see that the columns are highly correlated. This is expected considering that the video was obtained using a stationary camera and implies that the background of a video scene lies on a low-dimensional subspace. As a result, when Robust PCA is performed on the matrix M, the columns of the low-rank matrix L will correspond to the background of the frame and the columns of the matrix S will contain sparse deviations from the low-rank subspace. 

While the aforementioned description of Robust PCA operates on a video sequence in batch, there is also the newly proposed Grassmannian Robust Adaptive Subspace Tracking Algorithm (GRASTA) by He et al. \cite{GRASTA12} that learns the low-rank subspace by subsampling the video frames and proceeds in an online fashion. Although it is not Robust PCA in its purest sense, GRASTA still assumes that each frame can be represented as the sum of data generated from a low-dimensional subspace and a sparse error term.

In this paper, we will show how Robust PCA achieves superior performance to GMMs when applied to noisy videos. We will also describe how performing Robust PCA on each of the color channels of a video sequence and subsequent thresholding of the wavelet coefficients provide additional improvements. Should the user require a real-time alternative, we will also show how GRASTA also outperforms GMMs on the same noisy data. The remainder of this paper is organized as follows. Section 2 will describe the improvements made to the batch Robust PCA algorithm. Section 3 will present our experimental setup and findings. Section 4 will describe our conclusions and directions for future work.

\section{Method Description}
%\label{sec:format}

All printed material, including text, illustrations, and charts, must be kept
within a print area of 7 inches (178 mm) wide by 9 inches (229 mm) high. Do
not write or print anything outside the print area. The top margin must be 1
inch (25 mm), except for the title page, and the left margin must be 0.75 inch
(19 mm).  All {\it text} must be in a two-column format. Columns are to be 3.39
inches (86 mm) wide, with a 0.24 inch (6 mm) space between them. Text must be
fully justified.

\section{Experimental Results}
%\label{sec:pagestyle}

The paper title (on the first page) should begin 1.38 inches (35 mm) from the
top edge of the page, centered, completely capitalized, and in Times 14-point,
boldface type.  The authors' name(s) and affiliation(s) appear below the title
in capital and lower case letters.  Papers with multiple authors and
affiliations may require two or more lines for this information. Please note
that papers should not be submitted blind; include the authors' names on the
PDF.

\section{Conclusions}
\label{sec:typestyle}

To achieve the best rendering both in the proceedings and from the CD-ROM, we
strongly encourage you to use Times-Roman font.  In addition, this will give
the proceedings a more uniform look.  Use a font that is no smaller than nine
point type throughout the paper, including figure captions.

In nine point type font, capital letters are 2 mm high.  {\bf If you use the
smallest point size, there should be no more than 3.2 lines/cm (8 lines/inch)
vertically.}  This is a minimum spacing; 2.75 lines/cm (7 lines/inch) will make
the paper much more readable.  Larger type sizes require correspondingly larger
vertical spacing.  Please do not double-space your paper.  TrueType or
Postscript Type 1 fonts are preferred.

The first paragraph in each section should not be indented, but all the
following paragraphs within the section should be indented as these paragraphs
demonstrate.

% References should be produced using the bibtex program from suitable
% BiBTeX files (here: strings, refs, manuals). The IEEEbib.bst bibliography
% style file from IEEE produces unsorted bibliography list.
% -------------------------------------------------------------------------
\bibliographystyle{IEEEbib}
\bibliography{strings,refs}

\end{document}
